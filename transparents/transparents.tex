\documentclass[17 pt,french]{scrartcl}

\usepackage{amssymb, amsmath, mathtools}

\usepackage{fontspec}
\usepackage{xunicode}

\usepackage{amsfonts}
\usepackage{array}
\usepackage{listingsutf8}

\usepackage{lmodern}
\usepackage[top=1cm, bottom=1cm, left=1cm, right=1cm]{geometry} 
\usepackage{graphicx}
\usepackage{color}
\usepackage{fancyvrb}
\usepackage{tikz}
\usepackage{float}
\usepackage{wrapfig}
\usepackage{amsthm}
\usetikzlibrary{arrows,decorations.pathmorphing,backgrounds,fit,positioning,shapes.symbols,chains}

\definecolor{dkgreen}{rgb}{0,0.6,0}
\definecolor{gray}{rgb}{0.5,0.5,0.5}
\definecolor{mauve}{rgb}{0.58,0,0.82}
\DefineShortVerb{\|}

\lstset{language=Caml, 
		basicstyle=\fontsize{14}{15},
 		numbersep=5pt,             
 		backgroundcolor=\color{white}, 
 	    showspaces=false,               % show spaces adding particular underscores
 	    showstringspaces=false,         % underline spaces within strings
 	    showtabs=false,         
 	 	frame=single,                   % adds a frame around the code
 		tabsize=2,                      % sets default tabsize to 2 spaces
 		captionpos=b,                   % sets the caption-position to bottom
 		breaklines=true,                % sets automatic line breaking
 	 	breakatwhitespace=false,        
  		keywordstyle=\color{blue},        
 		commentstyle=\color{dkgreen},      
        stringstyle=\color{mauve},
        escapechar=!
		}

\newtheorem{prop}{Proposition}

\begin{document}
	\section*{Sommation formelle}
	
	On se donne une somme $S_n = \sum_{k=n_0}^n {a_k} $, existe-t-il une expression "fermée" de $S_n$, en fonction de n ?
	
    \subsection*{Exemples}
    	\begin{itemize}
    		\item $\sum_{k=0}^n k^2 = \frac{n(n+1)(2n+1)}{6}$\\
    		
		\item $\sum_{k=1}^n \frac{1}{k(k+1)} = \frac{n}{n+1}$\\
			
		\item $ \sum_{k=1}^n {\frac{n^3-2n^2-1}{n^4+n^2+1}(n-1)!} = \frac{n!}{n^2+n+1}$\\
		
		\item $ \sum_{k=1}^n {\frac{1}{n!}} = $ ? \\
		\item $ \sum_{k=1}^n {\frac{2n+3}{n(n+1)}} = $ ? \\
	\end{itemize}
	
	\subsection*{Formes fermées}
	On s'intéresse aux sommes  $S_n = \sum_{k=n_0}^n {a_k} $ telles que : $\frac{S_n}{S_{n-1}}$ soit une fraction rationnelle en n.
	
	\begin{prop}
	 $\frac{S_n}{S_{n-1}}$ est une fraction rationnelle en n ssi il existe deux fractions rationnelles $R(X)$ et $\alpha(X)$ telles que : 
	 \begin{itemize}
	 \item $\forall n \in \mathbb{N}$, $\frac{a_n}{a_{n-1}} = R(n)$
	 \item $\forall n \in \mathbb{N}$, $S_n = \alpha(n)a_n$
	 \end{itemize}
	\end{prop}

	Si c'est le cas on dit que $S_n$ admet une forme fermée.\\ \\
	\textbf{Algorithme de Gosper} :
		Etant donné la fraction rationnelle R(X) (determinée à l'aide de l'expression de $a_n$), l'algorithme doit déterminer si la fraction rationnelle $\alpha(X)$ existe et dans ce cas doit la calculer.
		\newpage
	\subsection*{Principe de l'algorithme}
	\begin{prop}
		Il existe 3 polynômes p(X), q(X), et r(X) tels que : 
		\begin{itemize}
		 \item $R(X) = \frac{p(X)q(X)}{p(X-1)r(X)}$
	 	 \item $\forall k \in \mathbb{N}$, $q(X) \wedge r(X+k) = 1 $
	 \end{itemize}
	\end{prop}
	\textbf{Preuve algorithmique} : \\
	On définit séquentiellement des polynômes $p_j(X)$, $q_j(X)$ et $r_j(X)$ tels que : $R(X) = \frac{p_j(X)q_j(X)}{p_j(X-1)r_j(X)}$ : 
		\begin{itemize}
		\item On définit : $p_0(X) = 1$ et $R(X) = \frac{q_0(X)}{r_0(X)} $ avec $q_0(X) \wedge r_0(X) = 1$ 
		\item Si $\forall k \in \mathbb{N}$, $q_j(X) \wedge r_j(X+k) = 1 $ on s'arrête
		\item Si $\exists k \in \mathbb{N}$, $q_j(X) \wedge r_j(X+k) \neq 1 $ : \\
		On pose $g(X)$ le pgcd de $q_j(X)$ et de $r_j(X+k)$ et : \\
		$q_{j+1}(X) = \frac{q_j(X)}{g(X)}$, $r_{j+1}(X) = \frac{r_j(X)}{g(X-k)}$ et \\$p_{j+1}(X) = p_j(X)g(X)g(X-1)...g(X-k+1)$
		
		On a $R(X) = \frac{p_{j+1}(X)q_{j+1}(X)}{p_{j+1}(X-1)r_{j+1}(X)}$ et deg $q_{j+1}(X) < $ deg $q_{j}(X)$
		\end{itemize}
		
		
\textbf{Problème} : \\
Comment tester algorithmiquement si :  $\forall k \in \mathbb{N}$, $q(X) \wedge r(X+k) = 1 $ ?
	
On introduit le \textbf{résultant} par rapport à la variable $X$ des polynômes $q(X)$ et $r(X+Y)$ : \\$G(Y) = Res(q(X), r(X+Y))$ 

On teste ensuite si ce polynôme G admet des \textbf{racines entières}.

	\begin{prop}
		Si la fraction rationnelle $\alpha(X)$ existe, alors elle est de la forme 
			$$\alpha(X) = \frac{q(X+1)}{p(X)}f(X)$$
	    avec f(X) polynôme vérifiant l'équation fonctionelle : 
	        $$p(X) = q(X+1)f(X)-f(X-1)r(X)$$
	\end{prop}
	
	\newpage
	\begin{prop}
	Majoration du degré de $f(X)$ : \\
	On pose $s_{+}(X) = q(X+1)+r(X)$, $s_{-}(X) = q(X+1)-r(X)$
	\begin{itemize}
	\item Si deg $s_{-}(X) \neq $ deg $s_{+}(X)-1$ alors 
		$$deg \ f(X) = deg \ p(X) - max(deg \ s_{-}(X),\ deg \ s_{+}(X) - 1)$$
    \item Si l = deg $s_{-}(X) = $ deg $s_{+}(X)-1$ \\
    	On écrit : $s_{-}(X) = u_lX^l + ... $ et $s_{+}(X) = v_{l+1}X^{l+1} + ... $ \\
    	On pose $n_0 = -2\frac{u_l}{u_{l+1}}$\\
    	Alors $deg \ f \leq  \left\{  \begin{array}{ll}
        deg \ p -l & \mbox{si } n_0 \notin \mathbb{N} \\
        max(deg \ p-l, n_0) & \mbox{sinon.}
    \end{array}
    \right.$
	\end{itemize}
	\end{prop}
	
	\subsection*{Exemple : $\sum_{k=0}^n k^2$}
	\begin{itemize}
	
\item	$a_n = n^2$, $\frac{a_n}{a_{n-1}} = \frac{n^2}{n^2-2n+1}$ donc $R(X) = \frac{X^2}{X^2-2X+1}$\\ \\
\item    On écrit $R(X) = \frac{p(X)q(X)}{p(X-1)r(X)}$ : on trouve $p(X) = X^2$, $q(X) = r(X) = 1$. \\
    
\item    On cherche f vérifiant (*) : $p(X) = q(X+1)f(X)-f(X-1)r(X)$.\\ \\
    La majoration donne f de degré au plus \textbf{3}.
    
\item    On cherche donc f sous la forme $f(X) = w_0 + w_1X+ w_2X^2+w_3X^3$
    
    (*) s'écrit $X^2 = (w_1 - w_2 + w_3) + (2 w_2-3w_3) X  + 3 w_3 X^2$ donc matriciellement : 
    $$
    \left(
  \begin{array}{ c c c c }
   0 & 1 & -1 & 1  \\
   0 & 0 & 2  & -3 \\
   0 & 0 & 0  & 3  \\
   0 & 0 & 0  & 0
  \end{array} \right) 
      \left(
  \begin{array}{ c }
     w_0  \\
     w_1  \\
     w_2  \\
     w_3
  \end{array} \right) = 
   \left(
    \begin{array}{ c }
     0  \\
     0  \\
     1  \\
     0
  \end{array} \right)
	$$

\item On trouve alors $f(X) =  1/3 X^3 +  1/2 X^2 +  1/6 X$.\\
Donc $\alpha(X) = \frac{q(X+1)}{p(X)}f(X) = f(X)/X^2 $ \\
Puis $S_n = n^2 \alpha(n) = 1/3 n^3 +  1/2 n^2 +  1/6 = \frac{n(n+1)(2n+1)}{6}$

\end{itemize}
\newpage
\subsection*{Programmation de l'algorithme en \textbf{Ocaml}}
	
	Programmation difficile car il faut en particulier : 
	\begin{itemize}
	\item Gérer les polynômes, les fractions rationelles sur $\mathbb{Q}$
	\item Gérer les matrices
	\item Calculer le résultant de deux polynômes de $\mathbb{(Q[X])[Y]}$
	\item Résoudre un système linéaire
	\item Trouver les racines entières d'un pôlynome
	\end{itemize}
	
	\textbf{Solution : } développer une librairie d'algèbre en Ocaml travaillant sur un corps quelconque. \\
	On utilise les fonctionnalités de \textbf{Modules} d'Ocaml : modules, modules de type et foncteurs.
	\subsubsection*{Définition d'un corps en Ocaml}
	\begin{lstlisting}
	module type DRing = sig
   	type elem
  	val zero: elem
  	val one : elem
   	val add : elem -> elem -> elem
   	val minus : elem -> elem
    val mult : elem -> elem -> elem
    val inv : elem -> elem
    val print : elem -> string
	end
	\end{lstlisting}
	\newpage
	\subsubsection*{Utilisation des foncteurs}
	On prend un module en paramètre pour créer un nouveau module : 
	\begin{lstlisting}
	module Matrix = functor (C : DRing) ->
  	struct
  	type matrix = C.elem array array
  	  (* ... *)
  	end
  	\end{lstlisting}
  	Librairie complète : 450 lignes de code.
  	\subsection*{Implémentation de l'algorithme à l'aide de cette librairie}
  	~ 175 lignes de code
  	\begin{lstlisting}
open Algebra
module C = DRing_Rat(* Les rationnels *)
module Q = Frac(DRing_Rat)(*Les fractions rat.*)
module Py = Poly(Q.DRing_F) (* Les fractions rationnelles à 2 variables*)

(* 1ère étape : on écrit R sous la forme : *)
(* R(X) = p(X)q(X) / p(X-1)r(X) avec pour tout k € N : q(X) ^ r(X+k) = 1 *)
(* Il faut donc calculer les polynomes p, q, r *)

(* 2ème étape : Majoration du degré de f *)

(* 3ème étape : calcul explicite de f(X) *)

let solve R = (* Calcul de alpha(X) à partir de R(X) *)
let solve_q a_n k0 = (* Calcul de sum_{k=k0}^n a_k si a_n est une fraction rationnelle en n *)
  	\end{lstlisting}
  	\newpage
  	\section*{Exemples}
  	 	\begin{itemize}
    		\item $\sum_{k=0}^n k^2$\\
    		\begin{lstlisting}
# solve_q ([((1, 1), 2)], [((1,1), 0)]) 0;;
p =  1x^2
q =  1
r =  1
Majoration du degré : 3
f =  1/6 x +  1/2 x^2 +  1/3 x^3
Résultat :  1/6 x +  1/2 x^2 +  1/3 x^3
    		\end{lstlisting}
    		On a donc $\sum_{k=0}^n k^2 = 1/6 n +  1/2 n^2 +  1/3 n^3 = \frac{n(n+1)(2n+1)}{6}$
    		
    		\item $\sum_{k=0}^n \frac{1}{(k+2)(k+5)}$\\
    		$(k+2)(k+5) = k^2+7k+10$ 
    		\begin{lstlisting}
# solve_q ([((1, 1), 0)], [((10,1),0); ((7,1), 1); ((1,1), 2)]) 0;;
p =  432 +  252x +  36x^2
q =  -1/6  +  -1/6 x
r =  -5/6  +  -1/6 x
Majoration du degré : 3
f =  3384 +  1728x +  216x^2
Résultat : ( 6 +  323/36 x +  10/3 x^2 +  13/36 x^3)/( 60 +  47x +  12x^2 +  1x^3)
    		\end{lstlisting}
    		On a donc $\sum_{k=0}^n \frac{1}{(k+2)(k+5)} = \frac{216+323 n+120 n^2+13 n^3}{36 \left(60+47 n+12 n^2+n^3\right)}$
    		
    	\newpage
    		\item $ \sum_{k=1}^n {\frac{n^3-2n^2-1}{n^4+n^2+1}(n-1)!}$ \\
    		$\frac{a_n}{a_{n-1}} = \frac{3-6 n+10 n^2-16 n^3+14 n^4-6 n^5+n^6}{-4+3 n-2 n^2+3 n^3-4 n^4+n^5} $ \\
    		\begin{lstlisting}
#let gR = [((3, 1), 0); ((-6, 1), 1); ((10, 1), 2); ((-16,1), 3); ((14,1), 4); ((-6,1),5); ((1,1), 6)], [((-4, 1), 0); ((3, 1), 1); ((-2,1),2); ((3,1), 3); ((-4,1), 4); ((1,1), 5)];;
print $ solve gR;;
    		
p =  -21/16  +  -21/8 x^2 +  21/16 x^3
q =  -16/7  +  32/7 x +  -64/21 x^2 +  16/21 x^3
r =  16/21  +  16/21 x +  16/21 x^2
Majoration du degré : d = 0
f =  441/256 
Résultat : ( 1x +  -1x^2 +  1x^3)/( -1 +  -2x^2 +  1x^3)- : unit = ()
    		\end{lstlisting}
    		Donc $S_n = a_n\frac{n - n^2 + n^3}{-1 - 2 n^2 + n^3} = \frac{n!}{1+n+n^2}$
    		
    		\item $\sum_{k=0}^n \frac{1}{k!}$\\
    		$\frac{a_n}{a_{n-1}} = \frac{1}{n}$\\
    		\begin{lstlisting}
# print $ solve (Q.normalise ([((1, 1), 0)], [((1, 1), 1)]));;
p =  1
q =  1
r =  1x
Majoration du degre : d = -1
Pas de solution.
    		\end{lstlisting}   		
    		\end{itemize}
	Il n'existe donc pas d'expression "fermée" de $\sum_{k=0}^n \frac{1}{k!}$.
	\section*{Calcul des racines entières d'un polynome rationnel}
	$a_0 + a_1X + ... + a_dX^d \in \mathbb{Z[X]}$\\
	Soit p $\in \mathbb{N}$ tel que $a_0 + a_1p + ... + a_dp^d = 0$\\
	Alors $a_0 = -p(a_1 + ... + a_np^{d-1})$\\
	Donc \textbf{p divise $a_0$}
	\section*{Preuve de la proposition 1 : }
	$\frac{S_n}{S_{n-1}}$ est une fraction rationnelle en n ssi il existe deux fractions rationnelles $R(X)$ et $\alpha(X)$ telles que : 
	 \begin{itemize}
	 \item $\forall n \in \mathbb{N}$, $\frac{a_n}{a_{n-1}} = R(n)$
	 \item $\forall n \in \mathbb{N}$, $S_n = \alpha(n)a_n$
	 \end{itemize}
	 \textbf{Preuve : }\\
	 \begin{itemize}
	 \item C'est suffisant : $\frac{S_n}{S_{n-1}} = \frac{\alpha(n)a_n}{\alpha(n-1)a_{n-1}} = \frac{\alpha(n)R(n)}{\alpha(n-1)}$ fraction rationnelle en n.
	 \item C'est nécessaire : supposons que $\frac{S_n}{S_{n-1}} = \sigma(n)$ est une fraction rationnelle en n.\\
	 On a $\frac{a_n}{a_{n-1}} = \frac{S_n-S_{n-1}}{S_{n-1}-S_{n-2}} = \frac{S_n-S_{n-1}}{S_{n-1}(1-\frac{S_{n-2}}{S_{n-1}})} = \frac{\sigma(n)-1}{1-\frac{1}{\sigma(n-1)}}$ \\
	 Donc $R(X) = \frac{\sigma(X)-1}{1-\frac{1}{\sigma(X-1)}}$\\
	 Et $a_n = S_n-S_{n-1} = S_n(1-\frac{1}{\sigma(n)})$ donc $S_n = \frac{1}{1-\frac{1}{\sigma(n)}}a_n$\\
	 Donc $\alpha(X) =  \frac{1}{1-\frac{1}{\sigma(X)}}$
	 \end{itemize}
	 
	 
	 
	 
	 
	 
	 
	\newpage
	\section*{Preuve de la proposition 3}
	\begin{itemize}
	\item On a $1 = \frac{S_n-S_{n-1}}{a_n} = \frac{\alpha(n)a_n-\alpha(n-1)a_{n-1}}{a_n} = \alpha(n) - \frac{\alpha(n-1)}{R(n)}$\\
	Donc $1 = \alpha(n)-\frac{\alpha(n-1)p(n-1)r(n)}{p(n)q(n)}$\\
	Donc $\forall n \in \mathbb{N}, p(n)q(n) = \alpha(n)p(n)q(n) - \alpha(n-1)p(n-1)r(n)$
	\item On pose $\beta(X) = \alpha(X)p(X)$\\
	On a donc $p(X)q(X) = \beta(X)q(X)-\beta(X-1)r(X)$
	\item $\beta$ est un polynôme !\\
	Supposons par l'absurde que ça ne soit pas le cas. Soit $z$ un pôle (complexe) de $\beta$.
		\begin{itemize}
		\item $p(X)q(X) = \beta(X)q(X)-\beta(X-1)r(X)$ donc : \\
		$z$ pôle complexe de $\beta(X-1)$ ou $q(z) = 0$
		\item $p(X+1)q(X+1) = \beta(X+1)q(X+1)-\beta(X)r(X+1)$ donc : \\
		$z$ pôle complexe de $\beta(X+1)$ ou $r(z+1) = 0$
		\item $\beta$ a un nombre fini de pôles, donc il existe $z$ : $z$ pôle de $\beta$ et $z-1$ n'est pas pôle de $\beta$.
		\item $z-1$ n'est pas pôle de $\beta$ donc $z$ n'est pas pôle de $\beta(X-1)$. \\
		Donc $q(z) = 0$.\\
		$\forall k \in \mathbb{N}$, $q(X) \wedge r(X+k) = 1 $, donc $\forall k \in \mathbb{N}, r(z+k) \neq 0$.\\
		Donc $z+1$ pôle de $\beta$ et par récurrence : $\forall k \in \mathbb{N}$ : $z+k$ pôle de $\beta$.
		\item Donc $\beta$ a un nombre infini de pôles, c'est absurde, donc $\beta$ est un polynôme.
		\end{itemize}
	\item $p(X+1)q(X+1) = \beta(X+1)q(X+1)-\beta(X)r(X+1)$ et \\$q(X+1)\wedge r(X+1) = 1$ donc $\beta(X) = f(X)q(X+1)$, f polynôme.
	On a alors :  $$p(X) = q(X+1)f(X)-f(X-1)r(X)$$ $$\alpha(X) = \frac{q(X+1)}{p(X)}f(X)$$
	\end{itemize}
\end{document}

